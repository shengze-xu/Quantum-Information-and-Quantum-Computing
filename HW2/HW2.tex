\documentclass[11pt]{article}
\usepackage{amssymb}
\usepackage{algorithm}  
\usepackage{algpseudocode}  
\usepackage{amsmath}  
\renewcommand{\algorithmicrequire}{\textbf{Input:}}  % Use Input in the format of Algorithm  
\renewcommand{\algorithmicensure}{\textbf{Output:}}
\parindent=22pt
\parskip=3pt
\oddsidemargin 18pt \evensidemargin 0pt
\leftmargin 1.5in
\marginparwidth 1in \marginparsep 0pt \headsep 0pt \topskip 20pt
\textheight 225mm \textwidth 148mm
\renewcommand{\baselinestretch}{1.15}
\begin{document}
\title{{\bf The Second Assignment}}
\author{3190102721 Xu Shengze}
\date{}
\maketitle

{\bf Note:} Some topics in this homework were completed after discussing with Zhou Yuxin.

\begin{tabular*}{13cm}{r}
\hline
\end{tabular*}

\vskip 0.3 in

{\bf Problem 1.1} Construct an algorithm that determines whether a given set of Boolean functions $\mathcal{A}$ constitues a complete basis. (Functions are represented by tables.) 

\vskip 0.3 in

{\bf Answer 1.1 }

According to the meaning of the question, all functions here are based on two variables. We consider formulating these functions on the basis of $\mathcal{A}$, then $\mathcal{A}$ is complete. Because functions are represented by tables, and there are four combinations of two variables, each of which has two values, the number of functions for two variables is $2^4=16$.

Next we need to construct an algorithm to check whether $\mathcal{A}$ contains these 16 functions. First, we need to check whether there are functions in $\mathcal{A}$ that can handle two variables $p_1(x,y)=x$ and $p_2(x,y)=y$. If the conclusion is negative, it means that $\mathcal{A}$ is incomplete. If it exists, the next step is for the set $\mathcal{F}$ of already constructed functions. We add to the set $\mathcal{F}$ all functions of the form $f(g_1(x_1,x_2),g_2(x_3,x_4),\cdots,g_k(x_{2k-1},x_{2k} ))$, where $x_j\in\{x,y\},g_j\in\mathcal{F},f\in\mathcal{F}$. If the set $\mathcal{F}$ increases, we repeat the procedure. Otherwise, there are two possibilities. One is that we have got all the functions in two variables when the algorithm stops, which shows that $\mathcal{A}$ is complete. Therefore, there is another possibility that $\mathcal{A}$ is incomplete.

\vskip 0.3 in
\newpage
{\bf Problem 2.2}
Let $c_n$ be the maximum complexity $c(f)$ for Boolean functions $f$ in $n$ variables. Prove that $1.99^n<c_n<2.01^n$

\vskip 0.3 in

{\bf Answer}

As for the upper bound, it's obvious that an upper bound is $O(n2^n)$. A circuit composed of n variables ends in at most $n$ steps, and each step has at most $2^n$ assignments. Hence the upper bound of the size is $O(n2^n)$ which is less than $2.01^n$ when $n$ is large enough.

As for the lower bound, we compare the number of Boolean functions in $n$ variables (obviously, we have $2^{2^n}$ functions) and the number of all circuits of a given size. Assume that the standard complete basis is used. For the $k$-th step of assignment of the circuit there are at most $O((n+k)^2)$ possibilities (two arguments can be chosen among $n+k-1$ variables, where there are $n$ input and $ k-1$ auxiliary variables, then the lower bound is $O(C_{n+k-1}^2)=O((n+k)^2)$.) Therefore, the number $N_s$ of different circuits of size s does not exceed
$$
O(((n+s)^2)^s)=2^{2s(\log(n+s)+O(1))}
$$
But the number of Boolean functions in $n$ variables equals $2^{2^n}$. If
$$
2^n>2s(\log(n+s)+O(1))
$$
there are more functions than circuits, so that $c_n>s$. If $s=1.99^n$, then the inequality above is satisfied for suffiiently large n.
\end{document}
